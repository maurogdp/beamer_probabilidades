\documentclass[10pt]{beamer}
\usetheme{Boadilla}
%\usetheme{Frankfurt}
\usecolortheme{whale}



%\usepackage{xcolor}
\xdefinecolor{lavanda}{rgb}{0.8,0.6,1}
\xdefinecolor{oliva}{cmyk}{0.64,0,0.95,0.4}
\xdefinecolor{minaranja}{rgb}{0.94,0.48,0.2}

%\usecolortheme[named=oliva]{structure}

\setbeamercovered{transparent}

\theoremstyle{plain} % default
\newtheorem{thm}{Teorema}
\newtheorem{conj}{Conjetura}
\newtheorem{prop}{Proposición}

\theoremstyle{definition}
\newtheorem{defn}{Definición}
\newtheorem{exmp}{Ejemplo}

\theoremstyle{remark}
\newtheorem{rem}{Nota}

\usepackage{ragged2e}



\usepackage[export]{adjustbox}

\newcommand{\dfn}[2]{
\begin{defn}[#1]
\justifying
#2
\end{defn}
}

\newcommand{\ejem}[2]{
\begin{exampleblock}{Ejemplo #1}
\justifying
#2
\end{exampleblock}
}

\newcommand{\ejemc}[1]{
\begin{exampleblock}{}
\justifying
#1
\end{exampleblock}
}
\newcommand{\sol}[1]{Solución:\\
#1
}

    \usepackage{stackengine}
\newcommand\xrowht[2][0]{\addstackgap[.5\dimexpr#2\relax]{\vphantom{#1}}}

\newcommand{\aca}[2]{\action<#1-|alert@#1>{#2}}
% first argument: slide number to appear from, second argument: content of header 
\newcommand{\ac}[2]{\action<#1->{#2}}
% first argument: slide number to appear from, second argument: content of cell


\title[Distribuciones de probabilidad]{Probabilidades}
\subtitle[Probabilidad y estadística]{Plan diferenciado - Probabilidad y estadística descriptiva e inferencial}
\author[Mauro Díaz]{Mauro Díaz Poblete}

\date{Agosto, 2023}
\institute[Manquemávida]{Departamento de matemática \\ Colegio Manquemávida}
\justifying
\begin{document}

%frame 1
\begin{frame}

\maketitle
\end{frame}


\begin{frame}
\frametitle{Índice}
\tableofcontents
\end{frame}



\section{Variables aleatorias discretas}

%frame 2
\begin{frame}
\justifying
\frametitle{Definición básica}

\dfn{Variable aleatoria discreta}{
Se dice que una variable aleatoria $X$ es discreta si puede tomar sólo un número finito o infinito numerable de valores distintos. 
}
\pause
\ejem{1}{
El número de bacterias por unidad de área en el estudio de control de medicamentos sobre el crecimiento de bacterias es una variable aleatoria
discreta, como lo es el número de televisores defectuosos en un envío de 100 aparatos. De hecho, las variables aleatorias discretas representan con frecuencia cantidades asociadas con
fenómenos reales.
}
Como ciertos tipos de variables aleatorias se presentan frecuentemente en la práctica, es útil tener a mano la probabilidad para cada valor de una variable aleatoria. Este conjunto de probabilidades recibe el nombre de \textit{distribución de probabilidad} de la variable aleatoria discreta.
\end{frame}

\begin{frame}
\justifying
\frametitle{La distribución de probabilidad para una variable aleatoria discreta}
\dfn{Función de probabilidad}{
La probabilidad de que $Y$ tome el valor $y$, $P(Y = y)$, se define como la suma de las probabilidades de todos los puntos muestrales en $\Omega$ a los que se asigna el valor $y$. A veces denotaremos $P(Y = y)$ por $p(y)$ o $f(y)$.
}
\pause
\dfn{distribución de probabilidad}{
La distribución de probabilidad para una variable discreta $Y$ puede ser representada por una fórmula, una tabla o una gráfica que produzca $p(y) = P(Y = y)$ para toda $y$.
}
\pause
Nótese que $p(y) \geq 0$ para toda y, pero la distribución de probabilidad para una variable 
aleatoria discreta asigna probabilidades diferentes de cero a sólo un número contable de valores y distintos. 
Se entiende que cualquier valor y no explícitamente asignado a una posibilidad positiva es tal que $p(y) = 0$.
\end{frame}

\begin{frame}
    \justifying
    \frametitle{La distribución de probabilidad para una variable aleatoria discreta}
    \ejem{2}{
        Un supervisor en una planta manufacturera tiene tres hombres y tres mujeres trabajando para 
        él y desea escoger dos trabajadores para un trabajo especial. No queriendo mostrar sesgo en 
        su selección, decide seleccionar los dos trabajadores al azar. Denote con $X$ el número de mujeres en su selección. 
        Encuentre la distribución de probabilidad para $X$.
    }
    \end{frame}



%frame 3
\begin{frame}
\frametitle{Definiciones básicas}
\begin{itemize}[<+->]
\justifying
\begin{defn}[Muestra]
Es el subconjunto de la población.
\end{defn}

\item Una \textbf{muestra} es cualquier subconjunto de esa población. Por ejemplo, los automóviles que pertenecen a los profesores del departamento de matemática.

\begin{defn}[Variable]
Es una característica de interés relacionada con cada elemento individual de una población o muestra.
\end{defn}

\item Una \textbf{variable} es el ``valor en pesos'' de cada automóvil individual.
\end{itemize}
\end{frame}



%frame 4
\begin{frame}
\frametitle{Definiciones básicas}
\begin{itemize}[<+->]
\justifying
\begin{defn}[Dato]
Es el valor de la variable asociada a un elemento de una población o muestra. Este valor puede ser un número, una palabra o un símbolo.
\end{defn}

\item Un \textbf{dato} es e valor en pesos de un automóvil en particular. El automóvil del profesor Andrés, por ejemplo esta valuado en 8.000.000 de pesos.

\begin{defn}[Datos]
Son el conjunto de valores que se obtienen de la variable a partir de cada uno de los elementos que pertenecen a la muestra.
\end{defn}

\item Los \textbf{datos} serían el conjunto de valores que corresponden a la muestra obtenida (8 millones, 7.5 millones, 11,3 millones, ...).
\end{itemize}
\end{frame}




%frame 5
\begin{frame}
\frametitle{Definiciones básicas}
\begin{itemize}[<+->]
\justifying
\begin{defn}[Experimento]
Es una actividad planeada cuyos resultados producen un conjunto de datos.
\end{defn}

\item El \textbf{experimento} serían los métodos que se aplican para seleccionar los automóviles que integran a la muestra y determinar el valor de cada automóvil de la muestra. El experimento podría realizarse preguntando a cada miembro de departamento, o de otras formas.

\begin{defn}[Parámetro]
Es un valor numérico que resume todos los datos de una población.
\end{defn}

\item El \textbf{parámetro} sobre el que se esta buscando información es el valor ``promedio'' de todos los automóviles de la población.
\end{itemize}
\end{frame}





%frame 6
\begin{frame}
\frametitle{Definiciones básicas}
\begin{itemize}[<+->]
\justifying
\begin{defn}[Estadístico]
Es un valor numérico que resume los datos de una muestra.
\end{defn}

\item El \textbf{estadístico} que se encontrará es el valor ``promedio'' de todos los automóviles de la muestra.

\item \textbf{Nota:} Si se toma una segunda muestra, quizá el conjunto de personas seleccionadas sería diferente, por ejemplo el departamento de inglés, y en consecuencia, el estadístico promedio se anticiparía para un valor diferente. No obstante, el valor promedio de ``todos los automóviles del profesorado'' no cambiaría.
\end{itemize}
\end{frame}

%frame 7
\begin{frame}
\frametitle{Definiciones básicas - Ejercicios}
\begin{itemize}
\justifying
\item[1.]De la población de adultos en Estados Unidos,
$36\%$ tiene una alergia. Una muestra de 1200 adultos
seleccionados al azar resultó que $33{,}2\%$ tenían algún
tipo de alergia.\\
\begin{itemize}
\item[a.]Describa a la población.
\item[b.]¿Cuál es la muestra?
\item[c.]Describa la variable.
\item[d.]Identifique el estadístico y dé su valor.
\item[e.]Identifique el parámetro y dé su valor.
\end{itemize}
\end{itemize} 
\end{frame}


%frame 8
\begin{frame}
\frametitle{Definiciones básicas - Ejercicios}
\begin{itemize}
\justifying
\item[2.]Un fabricante de medicamentos está interesado
en la proporción de personas que tienen hipertensión
(presión sanguínea elevada) considerando que esta
condición puede controlarse con un medicamento
nuevo que la compañía ha perfeccionado. Se efectúa
un estudio que abarca 5000 personas que padecen hipertensión,
y se encuentra que $80\%$ de ellas pueden
controlar su hipertensión con el medicamento. Suponiendo
que las 5000 personas sean representativas del
grupo que tiene hipertensión, conteste las siguientes
preguntas:\\

\begin{itemize}
\item[a.]¿Cuál es la población?
\item[b.]¿Cuál es la muestra?
\item[c.]Identifique el parámetro de interés.
\item[d.]Identifique el estadístico y dé su valor.
\item[e.]¿Conocemos el valor del parámetro?
\end{itemize}
\end{itemize} 
\end{frame}

\section{Tipos de variables}
%frame 9
\begin{frame}
\frametitle{Tipos de variables}
\begin{itemize}[<+->]
\justifying
\item Existen dos tipos de variables: 1) las variables que resultan de información \textit{cualitativa} y (2) variables que resultan de información \textit{cuantitativa}.

\begin{defn}[Variable cualitativa, de atributos, o categórica]
Es una variable que clasifica o describe a un elemento de una población.
\end{defn}
\item Una muestra de siete clientes de una peluquería fue cuestionada en cuanto al ``color de su cabello'', ``la ciudad donde vive'' y el ``nivel de satisfacción'' respecto a los resultados en la peluquería. Las tres variables son ejemplos de variables cualitativas (de atributos), ya que describen alguna característica de la persona, y todas las personas con el mismo atributo pertenecen a la misma categoría. Los datos recolectados fueron {rubio, café, negro, café}, {San Fernando, Rancagua, Santa Cruz, Teno}, y {muy satisfecho, satisfecho, algo satisfecho}.
\end{itemize}
\end{frame}


%frame 10
\begin{frame}
\frametitle{Tipos de variables}
\begin{itemize}[<+->]
\justifying
\begin{defn}[Variable cuantitativa o numérica]
Es una variable que cuantifica un elemento de una población.
\end{defn}

\item El ``costo total'' de los libros de texto adquiridos por cada estudiante para las clases
de este semestre es un ejemplo de variable cuantitativa (numérica). Se obtuvo
una muestra con los datos siguientes: $\$225.800$, $\$85.400$, $\$128.750$. (Para determinar
el ``costo promedio'', simplemente se suman los tres números y el resultado se divide
entre tres: $(225.800 + 85.400 + 128.750)/3 = 146.650$.)

\item \textbf{Nota:} Algunas operaciones aritméticas, como sumar y promediar, tienen sentido para los datos que resultan de una variable cuantitativa. 
\end{itemize}
\end{frame}




%frame 12
\begin{frame}
\frametitle{Tipos de variables - Variables cualitativas}
\begin{itemize}[<+->]
\justifying
\item Las variables cualitativas pueden caracterizarse como nominales u ordinales.

\begin{defn}[Variable nominal]
Es una variable cualitativa que caracteriza (describe o identifica) a un elemento de una población. Para los datos resultantes de una variable nominal, las operaciones aritméticas no solo carecen de sentido sino que tampoco se puede asignar un orden a las categorías.
\end{defn}
\item En la encuesta anterior que se aplicó a los cuatro clientes de una peluquería, dos de las variables, ``color de su cabello'' y ``ciudad donde vive'', son ejemplos de variables nominales, ya que ambas identifican alguna característica de la persona y carecerían de sentido para encontrar el promedio muestral al sumar y dividir entre cuatro. Por ejemplo, (rubio + café + negro + café)/4 no está definido. Además, el color del cabello y la ciudad donde vive no tienen un orden en sus categorías.
\end{itemize}
\end{frame}

%frame 13
\begin{frame}
\frametitle{Tipos de variables - Variables cualitativas}
\begin{itemize}[<+->]
\justifying

\begin{defn}[Variable ordinal]
es una variable cualitativa que presenta una posición, o clasificación, ordenada.
\end{defn}
\item En la encuesta anterior de cuatro clientes de una peluquería, la variable ``nivel de satisfacción'' es un ejemplo de variable ordinal, ya que presenta una clasificación ordenada: ``muy satisfecho'' está antes que ``satisfecho'', que se encuentra antes que ``algo satisfecho''. Otro ejemplo de una variable ordinal sería la clasificación de cinco fotografías de paisaje según la preferencia de alguien: primera elección, segunda elección, etcétera.
\end{itemize}
\end{frame}


%frame 14
\begin{frame}
\frametitle{Tipos de variables - Variables cuantitativas}
\begin{itemize}[<+->]
\justifying
\item Las variables cuantitativas o numéricas también pueden subdividirse en dos clasificaciones: variables discretas y variables continuas.
\begin{defn}[Variable discreta]
es una variable cuantitativa que puede asumir un número contable (o finito) de valores. Intuitivamente, la variable discreta puede asumir los valores correspondientes a puntos aislados a lo largo de un intervalo de recta. Es decir, entre dos valores cualesquiera siempre hay un hueco.
\end{defn}

\begin{defn}[Variable continua]
es una variable cuantitativa que puede asumir una cantidad incontable de valores. Intuitivamente, la variable continua puede asumir cualquier valor a lo largo de un intervalo de recta, incluyendo cualquier valor posible entre dos variables determinadas.
\end{defn}
\end{itemize}
\end{frame}

%frame 15
\begin{frame}
\frametitle{Tipos de variables - Variables cuantitativas}
En muchos casos, es posible distinguir los dos tipos de variables decidiendo si las variables están relacionadas con un conteo o una medición.
\begin{itemize}[<+->]
\justifying
\item La variable ``número de tíos que tiene cada uno de os estudiantes de este curso'' es un ejemplo de una variable discreta; sus valores se determina al contar el número de tíos.
\item La variable ``peso de los libros y material que carga al asistir hoy a clases'' es un ejemplo de variable aleatoria continua; los valores de la variable se encuentran midiendo el peso.
\end{itemize}
\end{frame}

%frame 16
\begin{frame}
\begin{alertblock}{Nota 1}
\justifying
Cuando intente determinar si una variable es continua o discreta, recuerde analizar la variable y piense en los valores que podrían ocurrir. No considere los valores de datos que se han registrado porque pueden ser engañosos.
\end{alertblock}
\begin{alertblock}{Nota 2}
\justifying
No permita que la apariencia de los datos lo engañe al momento de determinar el tipo de la variable. Las variables cualitativas no siempre son fáciles de reconocer, algunas veces se presentan como números. La muestra anterior de colores de cabello pueden codificarse como: 1 = negro, 2 = rubio, 3 = café. Los datos de la muestra se verían así: {2, 3, 1, 3} aunque siguen siendo datos de atributos.
\end{alertblock}

\end{frame}

%frame 17
\begin{frame}
\frametitle{Tipos de variables - Ejercicios}
\begin{itemize}
\justifying
\item[1.]El número de la camiseta de un equipo de futbol, ¿es una variable cuantitativa o es categórica? Apoye su respuesta con una explicación detallada.
\item[2.]En la actualidad se estudia la severidad de los efectos colaterales que experimentan ciertos pacientes cuando reciben tratamiento con un medicamento particular.
Esa severidad se mide en la escala: ninguna, benigna, moderada, grave, muy grave.\\

\begin{itemize}
\item[a.]Mencione la variable de interés.
\item[b.]Identifique el tipo de variable.

\end{itemize}
\item[3.]Al cuerpo docente de la universidad de Concepción se le hizo la siguiente pregunta ``¿Qué tan satisfecho estuvo usted con el programa de verano 2002?'' Sus respuestas fueron clasificadas como ``muy satisfecho'', ``un poco satisfecho'', ``ni satisfecho ni insatisfecho'', ``un poco insatisfecho'', o ``muy insatisfecho''.\\

\begin{itemize}
\item[a.]Mencione la variable de interés.
\item[b.]Identifique el tipo de variable.

\end{itemize}
\end{itemize} 
\end{frame}

%frame 18
\begin{frame}
\frametitle{Tipos de variables - Ejercicios}
\begin{itemize}
\justifying
\item[4.]Se pregunta a varios estudiantes el peso de los libros y demás material que llevan a clases.\\

\begin{itemize}
\item[a.]Identifique la variable de interés.
\item[b.]Identifique el tipo de variable.
\item[c.]Elabore una lista de algunos valores que podrían presentarse en una muestra.

\end{itemize}
\item[5.]Un técnico de control de calidad selecciona ciertas piezas ensambladas de una línea de montaje y registra la siguiente información sobre cada pieza:\\
A: defectuosa o no defectuosa\\
B: el número de identificación del trabajador que ensambló la pieza\\
C: el peso de la pieza\\
\begin{itemize}
\item[a.]¿Cuál es la población?
\item[b.]La población, ¿es finita o infinita?
\item[c.]¿Cuál es la muestra?
\item[d.]Clasifique las tres variables como cualitativas o cuantitativas.

\end{itemize}
\end{itemize} 
\end{frame}

%frame 19
\begin{frame}
\frametitle{Tipos de variables - Ejercicios}
\begin{itemize}
\justifying
\item[6.]Identifique las siguientes expresiones como ejemplos de variables (1) nominales, (2) ordinales, (3) discretas, o (4) continuas:\\

\begin{itemize}
\item[a.]Una encuesta de electores registrados donde se preguntaba a qué candidato daban su apoyo.
\item[b.]El tiempo necesario para que sane una herida cuando se aplica un nuevo medicamento.
\item[c.]El número de receptores de televisión en un hogar.
\item[d.]La distancia que viaja un balón de handball cuando es lanzado por las alumnas universitarias de primer año.
\item[e.]El número de páginas impresas que se procesan en la impresora de una computadora.

\end{itemize}
\end{itemize} 
\end{frame}

%frame 20
\begin{frame}
\frametitle{Medibilidad y variabilidad}
En un conjunto de datos siempre se espera variación. Si se encuentra poca variación o no se encuentra variación, podría suponerse que el instrumento de medición no está calibrado con una unidad que sea lo suficientemente pequeña.\\

Por ejemplo:\\
\begin{itemize}[<+->]
\justifying
\item Se toma un paquete de 24 barras de algún dulce favorito y cada barra se pesa en forma individual. Se observa que cada una de las 24 barras pesa 24.8 gramos, aproximado a la décima de gramo más próxima. ¿Significa esto que todas las barras pesan exactamente lo mismo?

\item No necesariamente. Suponga que las barras se pesan en una balanza analítica que registra la centésima de gramo más próxima. En este caso los pesos de las 24 barras mostrarían \textbf{variabilidad}.

\item No importa cuál sea la variable de respuesta: si la herramienta de medición es suficientemente exacta, habrá variabilidad en los datos. Uno de los objetivos primordiales del análisis estadístico es la medición de la variabilidad.

\end{itemize} 
\end{frame}

\section{Medibilidad y variabilidad}
%frame 21
\begin{frame}
\frametitle{Medibilidad y variabilidad - Ejercicios}
\begin{itemize}
\justifying
\item[1.]Suponga que trata de decidir la compra de una máquina entre dos opciones. Además, suponga que es importante la longitud a la que las máquinas cortan una pieza de un producto particular. Si ambas máquinas producen piezas de la misma longitud en promedio, ¿qué otra consideración sobre las longitudes sería importante?, ¿Por qué?

\item[2.]Durante años, grupos activistas de consumidores han pugnado por que los comerciantes al menudeo utilicen precios unitarios en sus productos. Argumentan que los precios de los alimentos, por ejemplo, siempre deben etiquetarse como $\$$/onza, $\$$/libra, $\$$/gramo, $\$$/litro, etc., además de estarlo como $\$$/paquete, $\$$/lata, $\$$/caja, $\$$/botella, etcétera. Explique por qué.


\end{itemize} 
\end{frame}

%frame 22
\begin{frame}
\frametitle{Medibilidad y variabilidad - Ejercicios}
\begin{itemize}
\justifying
\item[3.]Los profesores aplican exámenes para medir el grado de conocimiento de sus estudiantes acerca de su materia. Explique cómo es que ``una falta de variabilidad en las calificaciones de estudiantes podría indicar que el examen no fue una herramienta de medición muy eficaz''. Ideas a considerar: ¿Qué significaría si todos los estudiantes obtienen una calificación de $100\%$ en un examen? ¿Qué significaría si todos los estudiantes alcanzaran un $0\%$? ¿Qué significaría si las calificaciones varían de $40\%$ a $95\%$?
\end{itemize} 
\end{frame}

\section{Tablas de frecuencias}
%frame 23
\begin{frame}
\frametitle{Tablas de frecuencia}
Cuando sobre una población hemos realizado una encuesta o cualquier registro para conocer los valores que toman las variables, nos encontramos ante una gran cantidad de datos que debemos organizar. La mejor forma de organizar esta información es mediante tablas que llamaremos \textbf{tablas de frecuencias}.\\

\begin{defn}[Tabla de frecuencia]
Una tabla de frecuencia es una ordenación, en forma de tabla, de los datos recolectados, donde a cada dato se la asocia su frecuencia correspondiente 
\end{defn}
\end{frame}

%frame 24
\begin{frame}
\frametitle{Tablas de frecuencia - Tipos de frecuencia}
\begin{itemize}[<+->]
\justifying
\begin{defn}[Frecuencia absoluta]
La frecuencia absoluta es el número de veces que aparece un determinado valor en un estudio estadístico. Se representa por $f_i$.
\end{defn}
\item Supongamos una variable estadística $X$, constituida por $N$ valores, $X_1, X_2, X_3, \ldots, X_N$ procedentes de la observación de una determinada característica sobre una población o muestra compuesta por $N$ individuos. Y supondremos que toma k valores distintos que denotamos por $x_1, x_2, x_3, \ldots, x_k$ (con mayúscula todos los datos de la población y con minúscula los que son distintos).
\item La frecuencia absoluta $f_i$ es el número de veces que se repite el valor $x_i$.

\end{itemize} 
\end{frame}

%frame 25
\begin{frame}
\frametitle{Tablas de frecuencia - Tipos de frecuencia}
Estas frecuencias se disponen en forma de tabla, con la
siguiente estructura.
\begin{center}
\adjustbox{scale=0.7}{
\begin{tabular}{|c|c|}
\hline
\textbf{Dato} & \textbf{\begin{tabular}[c]{@{}c@{}}Frecuencia \\ absoluta ($f$)\end{tabular}} \\ \hline
$x_1$         & $f_1$                                                                       \\ \hline
$x_2$         & $f_2$                                                                       \\ \hline
$x_3$         & $f_3$                                                                       \\ \hline
$\ldots$      & $\ldots$                                                                    \\ \hline
$\ldots$      & $\ldots$                                                                    \\ \hline
$x_k$         & $f_k$                                                                       \\ \hline
\end{tabular}
}
\end{center}

\begin{itemize}[<+->]
\justifying
\item La suma de las frecuencias absolutas es igual al número total de datos, que se representa por $N$.
$$f_1+f_2+f_3+\ldots+f_k=N$$
\item Para indicar de manera más resumida estas sumas, se utiliza la letra griega $\Sigma$ (Sigma mayúscula) que se lee \textbf{``suma''} o \textbf{``sumatoria''}.
$$\sum_{i=1}^{k} f_i=N$$

\end{itemize} 
\end{frame}

%frame 26
\begin{frame}
\frametitle{Tablas de frecuencia - Tipos de frecuencia}
\begin{exampleblock}{Ejemplo}

\begin{itemize}[<+->]

\item A 60 alumnos de un colegio se les pregunta la edad, obteniendo los siguientes valores:\\

12; 13; 12; 12; 13; 14; 13; 13; 13; 12; 13; 14; 13; 15; 14; 13; 13; 13; 14; 14; 14; 15; 12; 15; 14; 15; 15; 16; 14; 16; 12; 14; 14; 14; 18; 15; 16; 16; 13; 15; 16; 14; 15; 17; 15; 16; 18; 16; 16; 16; 12; 14; 13; 13; 13; 16; 13; 16; 13; 12

\item Vamos a indicar en la columna ``dato'', los valores distintos que toma la variable, y en la columna ``frecuencia absoluta'' el número de veces que se repite cada valor.
\begin{center}
\adjustbox{scale=0.7}{
\begin{tabular}{|c|c|}
\hline
\textbf{Dato} & \textbf{\begin{tabular}[c]{@{}c@{}}Frecuencia \\ absoluta ($f$)\end{tabular}} \\ \hline
\aca{3}{12}&\aca{4}{8} \\ \hline
\aca{5}{13}&\aa{6}{16} \\ \hline
\aca{7}{14}&\aca{8}{13} \\ \hline
\aca{9}{15}&\aca{10}{9} \\ \hline
\aca{11}{16}&\aca{12}{11} \\ \hline
\aca{13}{17}&\aca{14}{1} \\ \hline
\aca{15}{18}&\aca{16}{2} \\ \hline
\end{tabular}
}
\end{center}
\end{itemize} 
\end{exampleblock}
\end{frame}

%frame 27
\begin{frame}
\frametitle{Tablas de frecuencia - Tipos de frecuencia}
\begin{itemize}[<+->]
\justifying
\begin{defn}[Frecuencia relativa]
La frecuencia relativa es el cociente entre la frecuencia absoluta de un determinado valor y el número total de datos. Se representa por $fr_i$ , aunque algunos autores la representan con $h_i$ o $n_i$.
\end{defn}
\item La expresión para calcular la frecuencia relativa es: $$fr_i=\frac{f_i}{N}$$
\item La suma de las frecuencias relativas es igual a 1. Es decir $\sum_{i=1}^{k} fr_i=1$

\item La frecuencia relativa también se puede expresar en forma de porcentajes:

$$p_i=100\cdot fr_i$$

A menudo se suele llamar a este valor como \textbf{``frecuencia relativa porcentual ($fr\%$)''}.
\end{itemize} 
\end{frame}

%frame 28
\begin{frame}
\frametitle{Tablas de frecuencia - Tipos de frecuencia}
\begin{itemize}[<+->]
\justifying
\begin{defn}[Frecuencia acumulada]
La frecuencia acumulada es la suma de las frecuencias absolutas de todos los valores inferiores o iguales al valor considerado. Se representa por $F_i$.
\end{defn}
\item Así, la frecuencia acumulada viene dada por la expresión:
$$F_j=\sum_{i_1}^{j}f_i$$

\item Es decir, $F_1=f_1$, $F_2=f_1+f_2$, y así sucesivamente hasta llegar a $F_k=f_1+f_2+f_3+...+f_k$

\item Por lo tanto, la frecuencia acumulada del último dato siempre será igual al número total de datos, $F_k=N$:

\end{itemize} 
\end{frame}

%frame 29
\begin{frame}
\frametitle{Tablas de frecuencia - Tipos de frecuencia}
Si completamos la tabla anterior con todas estas frecuencias, la tabla quedaría así:
\begin{center}
\adjustbox{scale=0.9}{
\begin{tabular}{|c|c|c|c|c|}
\hline
\textbf{Dato} & \textbf{\begin{tabular}[c]{@{}c@{}}Frecuencia \\ absoluta ($f$)\end{tabular}}& \textbf{\begin{tabular}[c]{@{}c@{}}Frecuencia \\ acumulada ($F$)\end{tabular}} &  \textbf{\begin{tabular}[c]{@{}c@{}}Frecuencia \\ relativa ($fr$)\end{tabular}}& \textbf{\begin{tabular}[c]{@{}c@{}}Frecuencia \\ porcentual ($f\%$)\end{tabular}} \\ \hline
$x_1$         & $f_1$ &$F_1=f_1$&$fr_1=\frac{f_1}{N}$& $p_1=100\cdot fr_1$                                                                       \\ \hline
$x_2$         & $f_2$ &$F_2=f_1+f_2$&$fr_2=\frac{f_2}{N}$& $p_2=100\cdot fr_2$                                                                      \\ \hline
$x_3$         & $f_3$ &$F_3=f_1+f_2+f_3$&$fr_3=\frac{f_3}{N}$& $p_3=100\cdot fr_3$                                                                      \\ \hline
$\ldots$         & $\ldots$ &$\ldots$&$\ldots$& $\ldots$                                                                      \\ \hline
$\ldots$         & $\ldots$ &$\ldots$&$\ldots$& $\ldots$                                                                      \\ \hline
$x_k$         & $f_k$ &$F_k=f_1+f_2+\ldots+f_k$&$fr_k=\frac{f_k}{N}$& $p_k=100\cdot fr_k$                                                                      \\ \hline
\end{tabular}
}
\end{center}
\end{frame}

%frame 30
\begin{frame}
\frametitle{Tablas de frecuencia - Tipos de frecuencia}
\begin{exampleblock}{Ejemplo}

Completamos la tabla de frecuencias del ejemplo anterior correspondiente al recuento de las edades de los 60 alumnos de un centro con los valores de las frecuencias relativas, porcentuales y acumuladas.
Vamos a hacer el recuento de los datos y presentarlos en una tabla de frecuencias:
\begin{center}
\adjustbox{scale=0.9}{
\begin{tabular}{|c|c|c|c|c|}
\hline
\textbf{Dato} & \textbf{\begin{tabular}[c]{@{}c@{}}Frecuencia \\ absoluta ($f$)\end{tabular}}& \textbf{\begin{tabular}[c]{@{}c@{}}Frecuencia \\ acumulada ($F$)\end{tabular}} &  \textbf{\begin{tabular}[c]{@{}c@{}}Frecuencia \\ relativa ($fr$)\end{tabular}}& \textbf{\begin{tabular}[c]{@{}c@{}}Frecuencia \\ porcentual ($f\%$)\end{tabular}} \\ \hline\xrowht{13pt}
$12$         & $8$ &\aca{2}{8}&\aca{3}{$\frac{8}{60}$}&\aca{4}{$13{,}\bar{3}\%$} \\ \hline \xrowht{13pt}
$13$         & $16$ &\aca{5}{24}&\aca{6}{$\frac{16}{60}$}&\aca{7}{$26{,}\bar{6}\%$}\\ \hline \xrowht{13pt}
$14$         & $13$ &\ac{8}{37}&\ac{9}{$\frac{13}{60}$}&\ac{10}{$21{,}\bar{6}\%$}\\ \hline \xrowht{13pt}
$15$         & $9$ &\ac{8}{46}&\ac{9}{$\frac{9}{60}$}&\ac{10}{$15\%$}\\ \hline \xrowht{13pt}
$16$         & $11$ &\ac{8}{57}&\ac{9}{$\frac{11}{60}$}&\ac{10}{$18{,}\bar{3}\%$}\\ \hline \xrowht{13pt}
$17$         & $1$ &\ac{8}{58}&\ac{9}{$\frac{1}{60}$}&\ac{10}{$1{,}\bar{6}\%$}\\ \hline \xrowht{13pt}
$18$         & $2$ &\ac{8}{60}&\ac{9}{$\frac{2}{60}$}&\ac{10}{$3{,}\bar{3}\%$}\\ \hline
\end{tabular}
}
\end{center}
\end{exampleblock}
\end{frame}

%frame 31
\begin{frame}
\frametitle{Tablas de frecuencia - Agrupación en intervalos}
\begin{itemize}[<+->]
\justifying
\item Cuando tenemos una variable que presenta una gran cantidad de datos agrupamos los valores en intervalos para realizar el recuento más fácilmente, trabajando así la variable como una variable continua. Los valores se agrupan usualmente en intervalos de la forma $\left(a, b\right]$.

\item Para establecer el número adecuado de intervalos hay varios métodos entre los que destacan: la fórmula de Sturges y la raíz del número de datos.
\begin{itemize}[<+->]
\item Número de intervalos $k=\sqrt{N}$, donde $N$ es el número total de datos.
\item \textbf{Fórmula de Sturges}: Número de intervalos $k=1+3{,}322\cdot \log (N)$, donde $N$ es el número total de datos.

\item Para poca cantidad de datos, aproximadamente menos de 50, el número de intervalos que entrega cada método es similar, sin embargo, para un número de datos muy grande, lo recomendable es utilizar la formula de Sturges, con la intención de que el número de intervalos no crezca demasiado.
\end{itemize}
\end{itemize}
\end{frame}

%frame 32
\begin{frame}
\frametitle{Tablas de frecuencia - Agrupación en intervalos}
\begin{itemize}
\justifying
\item Cuando ya hemos determinado el número de intervalos, los construimos. Generalmente los intervalos serán de la forma $(a_{i-1}, a_i]$ y, para construir la tabla de frecuencias, a cada uno de ellos se le asocia un valor representativo, denominado \textbf{marca de clase}, que se denota $c_i$, y que usualmente es el punto medio del intervalo, es decir:

$$c_i=\frac{a_{i-1}+a_i}{2}$$
\end{itemize}
\end{frame}

%frame 33
\begin{frame}
\frametitle{Tablas de frecuencia - Tipos de frecuencia}
\begin{exampleblock}{Ejemplo}
\begin{itemize}[<+->]

\item A 96 alumnos del colegio anterior se les pregunta la masa en kilogramos:\\
\vspace{0.5cm}
34,5; 35,2; 36,1; 37,0; 37,9; 38,5; 38,5; 39,1; 39,6; 40,0; 40,4; 40,4; 40,5; 40,8; 40,9; 41,1; 45,0; 45,2; 46,0; 47,3; 47,7; 47,8; 48,0; 48,2; 48,3; 48,3; 48,7; 49,0; 49,1; 49,1; 49,2; 50,3; 50,5; 50,5; 50,6; 50,9; 52,3; 52,8; 52,9; 53,0; 53,3; 53,5; 54,0; 54,2; 54,9; 55,1; 55,3; 55,3; 55,4; 55,6; 55,8; 55,8; 55,8; 56,0; 56,2; 56,4; 57,4; 58,1; 58,0; 58,9; 58,9; 59,0; 59,3; 59,3; 60,1; 60,4; 60,5; 60,5; 60,7; 62,5; 62,7; 63,0; 63,1; 63,2; 63,8; 64,6; 65,0; 65,0; 65,0; 65,5; 65,6; 65,7; 65,8; 68,2; 68,4; 69,6; 70,1; 70,3; 72,5; 72;5; 73,0; 79,0; 80,4; 80,7; 85,8; 108,4 

\end{itemize} 
\end{exampleblock}
\end{frame}

%frame 34
\begin{frame}
\frametitle{Tablas de frecuencia - Tipos de frecuencia}
\begin{exampleblock}{}
\begin{itemize}[<+->]
\item Primero, utilizaremos la fórmula de Sturges para determinar el número de intervalos que crearemos. 

$$1+3{,}322\cdot \log (N)=1+3{,}322\cdot \log 96\approx 7{,}58\approx 8$$

Por lo tanto, construiremos 8 intervalos. ($k=8$)\\

\item La siguiente parte consiste en determinar la amplitud de cada intervalo. Para esto calculamos el rango ($R$), esto es la diferencia entre el dato mayor y el dato menor. 

$$R=108{,}4-34{,}5=73{,}9\approx 74$$

\item Luego, la amplitud de la clase se calcula dividiendo el rango por el número de intervalos.

$$A=\frac{R}{k}= \frac{74}{8}=9{,}25$$

\end{itemize}
\end{exampleblock}
\end{frame}

%frame 35
\begin{frame}
\frametitle{Tablas de frecuencia - Tipos de frecuencia}
\begin{exampleblock}{}
Finalmete, podemos completar la columna de intervalos con la información encontrada y posteriormente la tabla por completo.

\begin{center}
\adjustbox{scale=0.9}{
\begin{tabular}{|c|c|c|c|c|}
\hline
\textbf{Intervalo} & \textbf{\begin{tabular}[c]{@{}c@{}}Frecuencia \\ absoluta ($f$)\end{tabular}}& \textbf{\begin{tabular}[c]{@{}c@{}}Frecuencia \\ acumulada ($F$)\end{tabular}} &  \textbf{\begin{tabular}[c]{@{}c@{}}Frecuencia \\ relativa ($fr$)\end{tabular}}& \textbf{\begin{tabular}[c]{@{}c@{}}Frecuencia \\ porcentual ($fr\%$)\end{tabular}} \\ \hline\xrowht{13pt}
\aca{2}{$[34.5, 43.75[$} &\ac{10}{16}  &\ac{11}{16}  &\ac{12}{$\frac{16}{96}$}&\ac{13}{$16{,}\bar{6}\%$}   \\ \hline\xrowht{13pt}
\aca{3}{$[43.75, 53[$} &\ac{10}{23}  &\ac{11}{39}  &\ac{12}{$\frac{23}{96}$}&\ac{13}{$23{,}958\bar{3}\%$}     \\ \hline\xrowht{13pt}
\aca{4}{$[53, 62.25[$} &\ac{10}{30}  &\ac{11}{69}  &\ac{12}{$\frac{30}{96}$}&\ac{13}{$31{,}25\%$}     \\ \hline\xrowht{13pt}
\aca{5}{$[62.25, 71.5[$} &\ac{10}{19}  &\ac{11}{88}  &\ac{12}{$\frac{19}{96}$}&\ac{13}{$19{,}791\bar{6}\%$}   \\ \hline\xrowht{13pt}
\aca{6}{$[71.5, 80.75[$} &\ac{10}{6}  &\ac{11}{94}  &\ac{12}{$\frac{6}{96}$}&\ac{13}{$6{,}25\%$}   \\ \hline\xrowht{13pt}
\aca{7}{$[80.75, 90[$} &\ac{10}{1}  &\ac{11}{95}  &\ac{12}{$\frac{1}{96}$}&\ac{13}{$1{,}041\bar{6}\%$}     \\ \hline\xrowht{13pt}
\aca{8}{$[90, 99.25[$} &\ac{10}{0}  &\ac{11}{95}  &\ac{12}{$\frac{0}{96}$}&\ac{13}{$\%$}     \\ \hline\xrowht{13pt}
\aca{9}{$[99.25, 108.5[$} &\ac{10}{1}  &\ac{11}{96}  &\ac{12}{$\frac{1}{96}$}&\ac{13}{$1{,}041\bar{6}\%$}  \\ \hline
\end{tabular}
}
\end{center}
\end{exampleblock}
\end{frame}

\section{Medidas de tendencia central}
%frame 36

\begin{frame}
\frametitle{Medidas de tendencia central}
\pause
Las \textbf{medidas de tendencia central} son valores numéricos que localizan, en algún sentido, el centro de un conjunto de datos. Es frecuente que el término promedio se asocie con todas las medidas de tendencia central.\\

\pause
Las medidas de tendencia central más utilizadas son las siguientes:

\begin{itemize}[<+->]
\item Media aritmética (o simplemente ``promedio'')
\item Media geométrica
\item Media armónica
\item Mediana
\item Moda
\end{itemize}

\ac{8}{Dependiendo la manera en que se presenten los datos, las formulas e interpretación de estas medidas cambian. Estas maneras puede ser:}
\begin{itemize}[<+->]
\item Datos no agrupados
\item Datos agrupados
\item Datos agrupados en intervalos
\end{itemize}
\end{frame}

\begin{frame}
\frametitle{Medidas de tendencia central - Datos no agrupados}
\begin{defn}[Media aritmética - Datos no agrupados]
La \textbf{media} de una muestra de valores $x_1, x_2, x_3, \ldots, x_n$ está dada por

$$\bar{x}=\frac{x_1+x_2+x_3+\ldots+x_{n-1}+x_n}{n}$$
o, de manera mas resumida, $\bar{x}=\frac{\sum_{i=1}^{n} x_i}{n}$

La media poblacional correspondiente se denota como $\mu$.
\end{defn}

\begin{alertblock}{Nota}

El símbolo $\bar{x}$, que se lee $x$ barra, se refiere a una media muestral. Por lo general no podemos medir el valor de la media poblacional, $\mu$; más bien, $\mu$ es una constante desconocida que podemos estimar usando información muestral.
\end{alertblock}

\end{frame}

\begin{frame}
\frametitle{Medidas de tendencia central - Datos no agrupados}
\begin{exampleblock}{Ejemplos - Media aritmética para datos no agrupados}
\begin{itemize}
\ac{1}{\item[1.] Un conjunto de datos consta de los cinco valores 6, 3, 8, 6 y 4. Encuentre la media.\\}
\ac{2}{\textbf{Solución:}\\
Con la fórmula dada, tenemos que

$$\bar{x}=\frac{\sum_{i=1}^{5} x_i}{5}=\frac{6+3+8+6+4}{5}=\frac{27}{5}=5{,}4$$}
\ac{3}{
Por lo tanto, la media aritmética de esta muestra es $5{,}4.$
}

\ac{4}{\item[2.]Un conjunto de datos consta de los cinco valores $0{,}06$, $0{,}03$, $0{,}08$, $0{,}06$ y $0{,}04$. Encuentre la media.\\}
\ac{5}{\textbf{Solución:}\\
Con la fórmula dada, tenemos que

$$\bar{x}=\frac{\sum_{i=1}^{5} x_i}{5}=\frac{0{,}06+0{,}03+0{,}08+0{,}06+0{,}04}{5}=\frac{0{,}27}{5}=0{,}054$$}
\ac{6}{
Por lo tanto, la media aritmética de esta muestra es $0{,}054.$}

\end{itemize}

\end{exampleblock}
\end{frame}







\begin{frame}
\frametitle{Medidas de tendencia central - Datos no agrupados}
\begin{alertblock}{Propiedades de la Media aritmética}
\begin{itemize}[<+->]
\item[1.] La media no tiene por qué ser igual a uno de los valores de los datos, ni siquiera de su misma naturaleza: datos enteros pueden tener una media decimal.
\item[2.] Si a todos los valores de la variable se le suma una misma cantidad, la media  aritmética queda aumentada en dicha cantidad.\\
Digamos que $MA(X)$ es la aplicación que calcula la media aritmética de la variable $X$, por lo tanto, $MA(X)=\bar{x}$. Dicho esto, la propiedad nos dice que:
$$MA(X+c)=MA(X)+MA(c)=\bar{x}+c \hspace{1cm}, \text{con $c$ constante.}$$
\item[3.] Si todos los valores de la variable se multiplican por una misma constante la media aritmética queda multiplicada por dicha constante. Es decir, 
$$MA(cX)=cMA(X)=c\bar{x} \hspace{1cm}, \text{con $c$ constante.}$$

\end{itemize}

\end{alertblock}
\end{frame}








\begin{frame}
\frametitle{Medidas de tendencia central - Datos no agrupados}
\begin{defn}[Media geométrica - Datos no agrupados]
La \textbf{media geométrica} de una muestra de valores $x_1, x_2, x_3, \ldots, x_n$ está dada por

$$\bar{x}=(x_1\cdot x_2\cdot x_3\cdot \ldots\cdot x_{n-1}\cdot x_n)^{\frac{1}{n}}=\sqrt[n]{x_1\cdot x_2\cdot x_3\cdot \ldots\cdot x_{n-1}\cdot x_n}$$
o, de manera mas resumida, $\bar{x}=\sqrt[n]{\Pi_{i=1}^{n} x_i}$

La media poblacional correspondiente se denota como $\mu$.
\end{defn}

\begin{alertblock}{Nota}

Siempre interpretaremos $\bar{x}$, como la media aritmética, a menos que se indique lo contrario.
\end{alertblock}

\end{frame}

\begin{frame}
\frametitle{Medidas de tendencia central - Datos no agrupados}
\begin{exampleblock}{Ejemplos - Media geométrica para datos no agrupados}
\begin{itemize}
\ac{1}{\item[1.] Un conjunto de datos consta de los cinco valores 6, 3, 8, 6 y 4. Encuentre la media geométrica.\\}
\ac{2}{\textbf{Solución:}\\
Con la fórmula dada, tenemos que

$$\bar{x}=\sqrt[5]{\Pi_{i=1}^{5} x_i}=\sqrt[5]{6\cdot 3\cdot 8\cdot 6\cdot 4}=\sqrt[5]{3456}\approx 5{,}1$$}
\ac{3}{
Por lo tanto, la media geométrica de esta muestra es aproximadamente $5{,}1.$
}
\end{itemize}

\end{exampleblock}
\end{frame}





\begin{frame}
\frametitle{Medidas de tendencia central - Datos no agrupados}
\begin{defn}[Media armónica - Datos no agrupados]
La \textbf{media armónica} de una muestra de valores $x_1, x_2, x_3, \ldots, x_n$ está dada por

$$\bar{x}=\frac{n}{\sum_{i=1}^{n}\frac{1}{x_i}}=\frac{n}{\frac{1}{x_1}+\frac{1}{x_2}+\ldots+\frac{1}{x_n}}$$
o, de manera mas resumida, $\bar{x}=\sqrt[n]{\Pi_{i=1}^{n} x_i}$

\end{defn}

\begin{alertblock}{Nota}

Siempre interpretaremos $\bar{x}$, como la media aritmética, a menos que se indique lo contrario.
\end{alertblock}

\end{frame}

\begin{frame}
\frametitle{Medidas de tendencia central - Datos no agrupados}
\begin{exampleblock}{Ejemplos - Media armónica para datos no agrupados}
\begin{itemize}
\ac{1}{\item[1.] Un conjunto de datos consta de los cinco valores 6, 3, 8, 6 y 4. Encuentre la media armónica.\\}
\ac{2}{\textbf{Solución:}\\
Con la fórmula dada, tenemos que

$$\bar{x}=\frac{5}{\sum_{i=1}^{5}\frac{1}{x_i}}=\frac{5}{\frac{1}{6}+\frac{1}{3}+\frac{1}{8}+\frac{1}{6}+\frac{1}{4}}=\frac{5}{\frac{25}{24}}=\frac{24}{5}=4{,}8$$}
\ac{3}{
Por lo tanto, la media armónica de esta muestra es aproximadamente $4{,}8.$
}
\end{itemize}

\end{exampleblock}
\end{frame}

















\begin{frame}
\frametitle{Medidas de tendencia central - Datos no agrupados}
\begin{defn}[Mediana - Datos no agrupados]
La mediana es el valor de los datos que ocupa la posición media cuando los datos
están clasificados en orden de acuerdo con su tamaño. La mediana la escribiremos como $Me$.\\
Partiendo de la base que los datos se encuentran ordenados ($x_1, x_2, x_3, x_4, \ldots, x_N$), la formula para calcular la media está dada por
$$M_e=\left\{
	       \begin{array}{ll}
		 \mathrm{Promedio\ entre\ }x_{\frac{N}{2}}\text{ y }x_{\frac{N}{2}+1}      & \mathrm{,\ si\ } N \mathrm{\ es\ par}\\
		 x_{\frac{N+1}{2}} & \mathrm{,\ si\ } N \mathrm{\ es\ impar}
		   \end{array}
	     \right.
	     $$
\aca{2}{No haremos distinción entre una mediana muestral y una poblacional.}
\end{defn}
\end{frame}

\begin{frame}
\frametitle{Medidas de tendencia central - Datos no agrupados}
\begin{exampleblock}{Ejemplos - Mediana para datos no agrupados}

\begin{itemize}
\ac{1}{\item[1.] Encuentre la mediana para el conjunto de datos $\{6, 3, 8, 5, 3\}$.\\}
\ac{2}{\textbf{Solución:}}\\
\aca{3}{Primero: ordenamos los datos en orden creciente o decreciente. $\{3,3,5,6,8\}$.}\\
\aca{4}{Segundo: como el número de datos es impar, la mediana ocupa la posición $\frac{N+1}{2}=\frac{5+1}{2}=3$.}\\
\aca{5}{Tercero: buscamos el dato que se encuentra en la posición 3. En este caso $x_3=5$.}\\
\aca{6}{Cuarto: la mediana esta dada por $M_e=5$}\\

\ac{7}{\item[2.]Encuentre la mediana de la muestra $\{9, 6, 7, 9, 10, 8\}$.\\}
\ac{8}{\textbf{Solución:}}\\
\aca{9}{Primero: ordenamos los datos en orden creciente o decreciente. $\{6,7,8,9,9,10\}$.}\\
\aca{10}{Segundo: como el número de datos es par, la mediana será el promedio entre los datos que se encuentran en las posiciones $\frac{N}{2}=\frac{6}{2}=3$ y $\frac{N}{2}+1=4$.}\\
\aca{11}{Tercero: buscamos los datos que se encuentran en las posiciones 3 y 4. En este caso $x_3=8$ y $x_4=9$.}\\
\aca{12}{Cuarto: la mediana esta dada por $M_e=\frac{8+9}{2}=\frac{17}{2}=8{,}5$.}
\end{itemize}

\end{exampleblock}
\end{frame}

\begin{frame}
\frametitle{Medidas de tendencia central - Datos no agrupados}
\begin{defn}[Moda - Datos no agrupados]
La moda es o los valores de los datos que se presenta con mayor frecuencia. La moda la escribiremos como $Mo$.
\end{defn}

\begin{exampleblock}{Ejemplos - Moda para datos no agrupados}
\begin{itemize}
\ac{1}{\item[1.] Encuentre la moda para el conjunto de datos $\{6, 3, 8, 5, 3\}$.\\}
\aca{2}{\textbf{Solución:} El dato que se presentan con mayor frecuencia en el conjunto es el 3, por lo tanto, $M_o=3$.}\\
\ac{3}{\item[2.] Encuentre la moda para el conjunto de datos $\{$café, rojo, verde, rojo, azul, café, rojo, azul, azul$\}$.}\\
\aca{4}{\textbf{Solución:} Los datos que se presentan con mayor frecuencia en el conjunto son ``rojo'' y ``azul'', ambos se presentan con igual frecuencia. En este caso decimos que el conjunto es \textbf{``bimodal''} y sus modas son ``rojo'' y ``azul''.}\\
\ac{5}{\item[3.] Encuentre la moda para el conjunto de datos $\{1,2,3,2,3,1,4,4\}$.\\}
\aca{6}{\textbf{Solución:} En este caso, no existe ningún dato que se repita más que otro, por lo tanto no posee moda. Decimos que el conjunto de datos es ``amodal''.}\\

\end{itemize}
\end{exampleblock}
\end{frame}







\begin{frame}
\frametitle{Medidas de tendencia central - Datos agrup. sin intervalos}
\begin{defn}[Media aritmética - Datos agrupados sin intervalos]
Supongamos que tenemos la siguiente tabla de frecuencias:
\begin{center}
\adjustbox{scale=0.8}{
\begin{tabular}{|c|c|c|c|c|}
\hline
\textbf{Dato} & \textbf{\begin{tabular}[c]{@{}c@{}}Frecuencia \\ absoluta ($f$)\end{tabular}}& \textbf{\begin{tabular}[c]{@{}c@{}}Frecuencia \\ acumulada ($F$)\end{tabular}} &  \textbf{\begin{tabular}[c]{@{}c@{}}Frecuencia \\ relativa ($fr$)\end{tabular}}& \textbf{\begin{tabular}[c]{@{}c@{}}Frecuencia \\ porcentual ($f\%$)\end{tabular}} \\ \hline
$x_1$         & $f_1$ &$F_1=f_1$&$fr_1=\frac{f_1}{N}$& $p_1=100\cdot fr_1$                                                                       \\ \hline
$x_2$         & $f_2$ &$F_2=f_1+f_2$&$fr_2=\frac{f_2}{N}$& $p_2=100\cdot fr_2$                                                                      \\ \hline
$x_3$         & $f_3$ &$F_3=f_1+f_2+f_3$&$fr_3=\frac{f_3}{N}$& $p_3=100\cdot fr_3$                                                                      \\ \hline
$\ldots$         & $\ldots$ &$\ldots$&$\ldots$& $\ldots$                                                                      \\ \hline
$\ldots$         & $\ldots$ &$\ldots$&$\ldots$& $\ldots$                                                                      \\ \hline
$x_k$         & $f_k$ &$F_k=f_1+f_2+\ldots+f_k$&$fr_k=\frac{f_k}{N}$& $p_k=100\cdot fr_k$                                                                      \\ \hline
\end{tabular}
}
\end{center}
Dada esta tabla de frecuencia, la media esta dada por
$$\bar{x}=\frac{\sum_{i=1}^{k} \left(x_i\cdot f_i\right)}{\sum_{i=1}^{k}f_i}$$

\end{defn}

\end{frame}

\begin{frame}
\frametitle{Medidas de tendencia central - Datos agrup. sin intervalos}
\begin{exampleblock}{Ejemplos - Media aritmética para datos agrupados sin intervalos}
\begin{itemize}
\ac{1}{\item[1.] Determina la media de los datos dados en la siguiente tabla de frecuencias.\\}
\begin{center}
\adjustbox{scale=0.8}{
\begin{tabular}{|c|c|}
\hline
\textbf{Dato} & \textbf{\begin{tabular}[c]{@{}c@{}}Frecuencia \\ absoluta ($f$)\end{tabular}} \\ \hline
$3$         & $2$ \\ \hline
$4$         & $3$ \\ \hline
$5$         & $8$ \\ \hline
$6$         & $7$ \\ \hline
$7$         & $6$ \\ \hline
$8$         & $1$ \\ \hline
\end{tabular}
}
\end{center}
\ac{2}{\textbf{Solución:}\\
Con la fórmula dada, tenemos que

$$
\begin{array}{ll}\xrowht{13pt}
		 \bar{x}=&\frac{\sum_{i=1}^{6} \left(x_i\cdot f_i\right)}{\sum_{i=1}^{6}f_i}=\frac{3\cdot 2+4\cdot 3+5\cdot 8+6\cdot 7+7\cdot 6+8\cdot 1}{2+3+8+7+6+1}\\ \xrowht{13pt}
		 &=\frac{6+12+40+42+42+8}{27}=\frac{150}{27}=5{,}\bar{5}
\end{array}
$$
}
\ac{3}{
Por lo tanto, la media aritmética de esta muestra es $5{,}\bar{5}.$
}

\end{itemize}

\end{exampleblock}
\end{frame}

\begin{frame}
\frametitle{Medidas de tendencia central - Datos agrup. sin intervalos}
\begin{defn}[Media geométrica - Datos agrupados sin intervalos]
La \textbf{media geométrica} está dada por la siguiente fórmula

$$\bar{x}=\sqrt[F_k]{\Pi_{i=1}^{k} x_i^{f_i}}$$

\end{defn}
\begin{defn}[Media armónica - Datos agrupados sin intervalos]
La \textbf{media armónica} está dada por la siguiente fórmula

$$\bar{x}=\frac{\sum_{i=1}^{k}f_i}{\sum_{i=1}^{k}\frac{f_i}{x_i}}$$

\end{defn}

\end{frame}

\begin{frame}
\frametitle{Medidas de tendencia central - Datos agrup. sin intervalos}
Para el caso de la mediana y la moda no existe una fórmula diferente para su cálculo, por lo tanto, solo ejemplificaremos el procedimiento.\\
\begin{exampleblock}{Ejemplos - Mediana para datos agrupados sin intervalos}

\ac{1}{Encuentre la mediana para el conjunto de datos ordenados en la siguiente tabla de frecuencias

\begin{center}
\adjustbox{scale=0.8}{
\begin{tabular}{|c|c|}
\hline
\textbf{Dato} & \textbf{\begin{tabular}[c]{@{}c@{}}Frecuencia \\ absoluta ($f$)\end{tabular}} \\ \hline
$12$         & $25$ \\ \hline
$13$         & $18$ \\ \hline
$14$         & $48$ \\ \hline
$15$         & $37$ \\ \hline
$16$         & $26$ \\ \hline
$17$         & $31$ \\ \hline
\end{tabular}
}
\end{center}

}
\ac{2}{\textbf{Solución:}}\\
\aca{3}{\textbf{Primero:} para facilitar el trabajo, es necesario completar la columna de frecuencia absoluta acumulada.}
\end{exampleblock}
\end{frame}







\begin{frame}
\frametitle{Medidas de tendencia central - Datos agrup. sin intervalos}
\begin{exampleblock}{Ejemplos - Mediana para datos agrupados sin intervalos}
\aca{1}{
\begin{center}
\adjustbox{scale=0.8}{
\begin{tabular}{|c|c|c|}
\hline
\textbf{Dato} & \textbf{\begin{tabular}[c]{@{}c@{}}Frecuencia \\ absoluta ($f$)\end{tabular}}&\textbf{\begin{tabular}[c]{@{}c@{}}Frecuencia \\ acumulada ($F$)\end{tabular}} \\ \hline
$12$         & $25$ & $25$\\ \hline
$13$         & $18$ & $43$\\ \hline
$14$         & $48$ & $91$\\ \hline
$15$         & $37$ & $128$\\ \hline
$16$         & $26$ & $154$\\ \hline
$17$         & $32$ & $186$\\ \hline
\end{tabular}
}
\end{center}
}
\aca{2}{\textbf{Segundo:} como la cantidad de datos es par, sabemos que los datos centrales ocuparan las posiciones $\frac{186}{2}=93$ y $94$.}\\
\aca{3}{\textbf{Tercero:} buscamos los datos que se encuentra en las posiciones 93 y 94. En este caso, podemos observar en la columna de frecuencia absoluta acumulada que, si escribiéramos todos los datos en fila, cuando escribamos el último dato 14, llevaríamos un total de 91 datos, por lo tanto, los primeros datos 15 que escribamos ocuparan las posiciones 92,93,94,95, $\ldots$, 128.}\\
\aca{4}{\textbf{Cuarto:} la mediana esta dada por $M_e=15$}\\
\end{exampleblock}
\end{frame}






\begin{frame}
\frametitle{Medidas de tendencia central - Datos agrup. sin intervalos}
Para el caso de la mediana y la moda no existe una fórmula diferente para su cálculo, por lo tanto, solo ejemplificaremos el procedimiento.\\
\begin{exampleblock}{Ejemplos - Moda para datos agrupados sin intervalos}

\ac{1}{Encuentre la moda para el conjunto de datos ordenados en la siguiente tabla de frecuencias

\begin{center}
\adjustbox{scale=0.8}{
\begin{tabular}{|c|c|}
\hline
\textbf{Dato} & \textbf{\begin{tabular}[c]{@{}c@{}}Frecuencia \\ absoluta ($f$)\end{tabular}} \\ \hline
$12$         & $25$ \\ \hline
$13$         & $18$ \\ \hline
$14$         & $48$ \\ \hline
$15$         & $37$ \\ \hline
$16$         & $26$ \\ \hline
$17$         & $31$ \\ \hline
\end{tabular}
}
\end{center}

}
\ac{2}{\textbf{Solución:}}\\
\aca{3}{La moda no tiene mayor dificultad para este caso. Simplemente debemos identificar el dato que posee la mayor frecuencia. $M_o=14$}
\end{exampleblock}
\end{frame}
































\begin{frame}
\frametitle{Medidas de tendencia central - Datos agrup. en intervalos}
\begin{defn}[Media, mediana y moda - Datos agrup. en intervalos]
Supongamos que tenemos la siguiente tabla de frecuencias:
\ac{2}{
\begin{center}
\adjustbox{scale=0.8}{
\begin{tabular}{|c|c|c|c|c|c|}
\hline
\textbf{\begin{tabular}[c]{@{}c@{}}Marca \\ de clase ($c$)\end{tabular}} & \textbf{Intervalo} & \textbf{\begin{tabular}[c]{@{}c@{}}Frecuencia \\ absoluta ($f$)\end{tabular}}& \textbf{\begin{tabular}[c]{@{}c@{}}Frecuencia \\ acumulada ($F$)\end{tabular}} \\ \hline
$c_1$&$I_1$         & $f_1$ &$F_1=f_1$\\ \hline
$c_2$&$I_2$         & $f_2$ &$F_2=f_1+f_2$\\ \hline
$c_3$&$I_3$         & $f_3$ &$F_3=f_1+f_2+f_3$\\ \hline
$\ldots$&$\ldots$         & $\ldots$ &$\ldots$ \\ \hline
$\ldots$&$\ldots$         & $\ldots$ &$\ldots$ \\ \hline
$c_k$&$I_k$         & $f_k$ &$F_k=f_1+f_2+\ldots+f_k$\\ \hline
\end{tabular}
}
\end{center}
}
\aca{3}{\textbf{Media.}\\ Dada esta tabla de frecuencia, la media esta dada por una formula similar a la anterior, la única diferencia es que ahora la marca de clase la consideremos como el dato representativo del intervalo
$$\bar{x}=\frac{\sum_{i=1}^{k} \left(c_i\cdot f_i\right)}{\sum_{i=1}^{k}f_i}$$
}
\end{defn}

\end{frame}


\begin{frame}
\frametitle{Medidas de tendencia central - Datos agrup. en intervalos}
\begin{defn}[]
\aca{1}{
\textbf{Mediana.}\\ Dada esta tabla de frecuencia, el valor aproximado de la mediana esta dada por la formula
$$M_e=L_{M_e}+\left(\frac{\frac{N}{2}-F_{i-1}}{f_i}\right)\cdot A$$
}
\aca{2}{Donde:\\
$L_{M_e}$: Es el límite inferior del intervalo que contiene la mediana.}\\
\aca{3}{$N$: es el total de datos.}\\
\aca{4}{$F_{i-1}$: Es la frecuencia acumulada hasta el intervalo anterior al que contiene la mediana.}\\
\aca{5}{$f_i$: Es la frecuencia absoluta del intervalo que contiene la mediana.}\\
\aca{6}{$A$: Es la amplitud del intervalo que contiene la mediana.}

\end{defn}

\end{frame}


\begin{frame}
\frametitle{Medidas de tendencia central - Datos agrup. en intervalos}
\begin{defn}[]
\aca{1}{
\textbf{Moda.}\\ 
Primero, para dar la siguiente fórmula, definiremos como \textbf{``intervalo modal''} al intervalo que posea la mayor frecuencia absoluta.\\

Dada la tabla de frecuencia, el valor aproximado de la moda esta dada por la formula
$$M_o=L_{M_o}+\left(\frac{D_A}{D_B+D_A}\right)\cdot A$$
}
\aca{2}{Donde:\\
$L_{M_o}$: Es el límite inferior del intervalo modal.}\\
\aca{3}{$D_A$: : Es la diferencia entre la frecuencia del intervalo modal y la clase anterior.}\\
\aca{4}{$D_B$: Es la diferencia entre la frecuencia del intervalo modal y la clase siguiente.}\\
\aca{6}{$A$: Es la amplitud del intervalo modal.}
\end{defn}

\end{frame}

\begin{frame}
\frametitle{Medidas de tendencia central - Datos agrup. en intervalos}
\begin{exampleblock}{Ejemplo - Media, mediana y moda en datos agrup. en intervalos}
Determina los valores aproximados de la media, la mediana y la moda dada la siguiente tabla de frecuencias.
\ac{1}{
\begin{center}
\adjustbox{scale=0.8}{
\begin{tabular}{|c|c|c|c|c|c|}
\hline
\textbf{\begin{tabular}[c]{@{}c@{}}Marca \\ de clase ($c$)\end{tabular}} & \textbf{Intervalo} & \textbf{\begin{tabular}[c]{@{}c@{}}Frecuencia \\ absoluta ($f$)\end{tabular}}& \textbf{\begin{tabular}[c]{@{}c@{}}Frecuencia \\ acumulada ($F$)\end{tabular}} \\ \hline\xrowht{13pt}
&$[5,5.5[$         & $15$ &\\ \hline\xrowht{13pt}
&$[5.5, 6[$         & $27$ &\\ \hline\xrowht{13pt}
&$[6, 6.5[$         & $35$ &\\ \hline\xrowht{13pt}
&$[6.5, 7[$         & $43$ &\\ \hline\xrowht{13pt}
&$[7, 7.5[$         & $80$ &\\ \hline\xrowht{13pt}
&$[7.5, 8[$         & $11$ &\\ \hline\xrowht{13pt}
&$[8, 8.5[$         & $59$ &\\ \hline
\end{tabular}
}
\end{center}
}
\end{exampleblock}
\end{frame}



\begin{frame}
\frametitle{Medidas de tendencia central - Datos agrup. en intervalos}
\begin{exampleblock}{}
Primero completamos la tabla de frecuencias.
\ac{2}{
\begin{center}
\adjustbox{scale=0.8}{
\begin{tabular}{|c|c|c|c|c|c|}
\hline
\textbf{\begin{tabular}[c]{@{}c@{}}Marca \\ de clase ($c$)\end{tabular}} & \textbf{Intervalo} & \textbf{\begin{tabular}[c]{@{}c@{}}Frecuencia \\ absoluta ($f$)\end{tabular}}& \textbf{\begin{tabular}[c]{@{}c@{}}Frecuencia \\ acumulada ($F$)\end{tabular}} \\ \hline\xrowht{13pt}
\aca{2}{$5.25$}&$[5,5.5[$         & $15$ &\aca{3}{$15$}\\ \hline\xrowht{13pt}
\aca{2}{$5.75$}&$[5.5, 6[$         & $27$ &\aca{3}{$42$}\\ \hline\xrowht{13pt}
\aca{2}{$6.25$}&$[6, 6.5[$         & $35$ &\aca{3}{$77$}\\ \hline\xrowht{13pt}
\aca{2}{$6.75$}&$[6.5, 7[$         & $43$ &\aca{3}{$120$}\\ \hline\xrowht{13pt}
\aca{2}{$7.25$}&$[7, 7.5[$         & $80$ &\aca{3}{$200$}\\ \hline\xrowht{13pt}
\aca{2}{$7.75$}&$[7.5, 8[$         & $11$ &\aca{3}{$211$}\\ \hline\xrowht{13pt}
\aca{2}{$8.25$}&$[8, 8.5[$         & $59$ &\aca{3}{$270$}\\ \hline
\end{tabular}
}
\end{center}
}
\end{exampleblock}
\end{frame}

\begin{frame}
\frametitle{Medidas de tendencia central - Datos agrup. en intervalos}
\begin{exampleblock}{}
\textbf{Media.}\\
\aca{2}{Aplicando la formula de la media, obtenemos que}
\aca{3}{
$$\begin{array}{ll}\xrowht{13pt}
\bar{x}&=\frac{5{,}25\cdot 15+5{,}75\cdot 27+6{,}25\cdot 35+6{,}75\cdot 43+7{,}25\cdot 80+7{,}75\cdot 11+8{,}25\cdot 59}{270}\\ \xrowht{13pt}
\bar{x}&=\frac{1895}{270}\\ \xrowht{13pt}
\bar{x}&\approx 7{,}02
\end{array}
$$
}
\aca{4}{\textbf{Mediana.}}\\
\aca{5}{Aplicando la formula de la mediana, obtenemos que}\\
\aca{6}{
$$\begin{array}{ll}\xrowht{13pt}
M_e&=7+\left(\frac{\frac{270}{2}-120}{80}\right)\cdot 0{,}5\\ \xrowht{13pt}
M_e&=7+\frac{15}{80}\cdot 0{,}5\\ \xrowht{13pt}
M_e&=7+\frac{3}{32}\\ \xrowht{13pt}
M_e&=7{,}09
\end{array}
$$
}
\end{exampleblock}
\end{frame}

\begin{frame}
\frametitle{Medidas de tendencia central - Datos agrup. en intervalos}
\begin{exampleblock}{}
\textbf{Moda.}\\
\aca{2}{El intervalo modal corresponde al intervalo $[7, 7{.}5[$.}\\
\aca{3}{Aplicando la formula de la media, obtenemos que}
\aca{4}{
$$\begin{array}{ll}\xrowht{13pt}
M_o&=7+\left(\frac{37}{69+37}\right)\cdot 0{,}5\\ \xrowht{13pt}
M_o&=7+\frac{37}{106}\cdot 0{,}5\\ \xrowht{13pt}
M_o&\approx 7{,}007
\end{array}
$$
}
\end{exampleblock}
\end{frame}

\begin{frame}
\frametitle{Ejercicios - Medidas de tendencia central}
\begin{itemize}
\item[1.]La tabla adjunta indica el número de ensayos que han desarrollado durante el verano los alumnos que rendirán la PAES. Determina la media, y los valores estimados de la mediana y la moda.\\

\begin{center}
\begin{tabular}{|c|c|}\hline
Intervalo & Frecuencia\\ \hline
$[2-6[$&$14$\\ \hline
$[6-10[$&$25$\\ \hline
$[10-14[$&$20$\\ \hline
$[14-18[$&$25$\\ \hline
$[18-22[$&$16$\\ \hline

\end{tabular}
\end{center}
\end{itemize}
\end{frame}

\begin{frame}
\frametitle{Medidas de dispersión}
Las medidas de dispersión describen la cantidad de dispersión, o variabilidad, que se encuentra entre los datos: los datos agrupados de manera estrecha tienen valores relativamente pequeños, y aquellos datos que estén más dispersos tienen valores más grandes. No hay límite sobre qué tan dispersos puedan ser los datos; por tanto, las medidas de dispersión pueden ser muy grandes. La medida de dispersión más sencilla es el rango. 
\begin{defn}[Rango]
El Rango, $R$, es la diferencia en valor entre el o los datos de valor más alto, $H$, y el o los datos de valor más bajo, $L$:
$$R=H-L$$
\end{defn}
\begin{exampleblock}{Ejemplo}
La muestra 3, 4, 5, 6, 8 tiene un rango de H – L = 8 – 3 = 5. El rango de 5 nos dice que estos datos caen todos ellos dentro de un intervalo de 5 unidades.
\end{exampleblock}
\end{frame}

\begin{frame}
\frametitle{Medidas de dispersión}
Las otras medidas de dispersión que se van a estudiar en este capítulo son medidas de dispersión alrededor de la media. Para desarrollar una medida de dispersión alrededor de la media, contestemos primero a la pregunta: ¿qué tan lejos de la media está cada x?
\begin{defn}[Desviación desde la media]
Una desviación desde la media, $(x – \bar{x} )$, es la diferencia
entre el valor de $x$ y la media, $\bar{x}$.
\end{defn}
\begin{exampleblock}{Ejemplo}
Considere la muestra 6, 3, 8, 5, 3. Usando la fórmula $\bar{x} =\frac{\sum x}{n}
$, encontramos
que la media es 5. Cada desviación, $(x – \bar{x})$, se encuentra entonces al restar 5 de
cada valor $x$:

\begin{center}
\begin{tabular}{l|ccccc}
Datos, $x$&6&3&8&5&3\\ \hline
Desviación, $x-\bar{x}$&1&$-2$&3&0&$-2$\\
\end{tabular}
\end{center}
\end{exampleblock}
\end{frame}


\begin{frame}
\frametitle{Medidas de dispersión}
Las otras medidas de dispersión que se van a estudiar en este capítulo son medidas de dispersión alrededor de la media. Para desarrollar una medida de dispersión alrededor de la media, contestemos primero a la pregunta: ¿qué tan lejos de la media está cada x?
\begin{defn}[Desviación media absoluta]
Es la media de los valores absolutos de las desviaciones desde la media:

$$\text{deviación media amsoluta}=\frac{\sum |x-\bar{x}|}{n}$$
\end{defn}
\begin{exampleblock}{Ejemplo}
Considere la muestra 6, 3, 8, 5, 3. Usando la fórmula $\bar{x} =\frac{\sum x}{n}
$, encontramos
que la media es 5. Luego,

$$
\begin{array}{rl}
\text{Des. media abs.}=&\frac{\sum_{i=1}^5|x_i-5|}{5}=\frac{|1|+|-2|+|3|+|0|+|-2|}{5}\\
=&\frac{1+2+3+0+2}{5}=\frac{8}{5}=1{,}6
\end{array}
$$


\end{exampleblock}
\end{frame}

\begin{frame}
\frametitle{Medidas de dispersión}
Una segunda forma de eliminar el efecto neutralizador positivo-negativo es elevar al cuadrado cada una de las desviaciones; el cuadrado de las desviaciones será de valores positivos (positivos o cero). El cuadrado de las desviaciones se usa para hallar la varianza.
\begin{defn}[Varianza muestral]
La varianza muestral, $s^2$, es la media del cuadrado de las desviaciones, calculada usando $n – 1$ como divisor:

$$s^2=\frac{\sum_{i=1}^n(x_i-\bar{x})^2}{n-1}$$

donde n es el tamaño muestral, es decir, el número de datos de la muestra.
\end{defn}

Si se conociera el valor de $\mu$, entonces se podría definir la varianza muestral como el promedio de las desviaciones cuadradas de las $x_i$ respecto a $\mu$. Sin embargo, el valor de $\mu$ casi nunca es conocido, por lo que se debe utilizar el cuadrado de la suma de las desviaciones respecto a $\bar{x}$. Pero las $x_i$ tienden a acercarse más a su valor promedio $\bar{x}$ que el promedio poblacional $\mu$. Para compensar lo anterior se utiliza el divisor $n — 1$ en lugar de $n$.

\end{frame}

\begin{frame}
\frametitle{Medidas de dispersión}
\begin{exampleblock}{Ejemplo}
La varianza de la muestra 6, 3, 8, 5, 3 se calcula a continuación
\begin{itemize}
\item Primero calculamos la media de los datos. En este caso, $\bar{x}=5$.
\item Calculamos las diferencias de los datos respecto a la media.
$$6-5=1\hspace{0.8cm} 3-5=-2\hspace{0.8cm} 8-5=3\hspace{0.8cm} 5-5=0\hspace{0.8cm} 3-5=-2$$
\item Calculamos los cuadrados de las diferencias.
$$1^2=1\hspace{1cm} (-2)^2=4\hspace{1cm} 3^2=9\hspace{1cm} 0^2=0\hspace{1cm} (-2)^2=4$$
\item Luego reemplazamos estos resultados en la fórmula
$$s^2=\frac{\sum_{i=1}^5 (x_i-5)^2}{5-1}=\frac{1+4+9+0+4}{4}=\frac{18}{4}=4{,}5$$
\end{itemize}
\end{exampleblock}
\end{frame}

\begin{frame}
\frametitle{Medidas de dispersión}
\begin{defn}[Desviación estandar muestral]
La desviación estándar de una muestra, $s$, es la raíz cuadrada de la varianza muestral:
$$s=\sqrt{s^2}=\sqrt{\frac{\sum_{i=1}^n(x_i-\bar{x})^2}{n-1}}$$
\end{defn}
\begin{exampleblock}{Ejemplo}
Calculemos la desviación estandar de la muestra 1, 3, 5, 6, 10.
\begin{itemize}
\item Primero calculamos su varianza

$$s^2=11{,5}$$

\item finalmente calculamos la raíz cuadrada de la varianza

$$s=\sqrt{s^2}=\sqrt{11{,}5}\approx 3{,4}$$
\end{itemize}
\end{exampleblock}

\end{frame}

\begin{frame}
\frametitle{Medidas de dispersión}
\begin{alertblock}{Nota}
Es frecuente que el numerador para la varianza muestral, $\sum (x – \bar{x})^2$, se denomine suma de los cuadrados de $x$ y se simbolice por $SS(x)$. Así, la fórmula de la varianza de puede expresar como

$$s^2=\frac{SS(x)}{n-1}$$


\end{alertblock}

\end{frame}

\begin{frame}
\frametitle{Medidas de dispersión}
Los cálculos necesarios para determinar el valor de la varianza no siempre son fáciles de realizar. Sin embargo, La ``suma de los cuadrados de $x$'' se puede reescribir de tal manera que no sea necesario calcular la media y así simplificar en cierto grado el procedimiento.\\

\begin{prop}[Suma de los cuadrados de $x$]

$$SS(x)=\sum x-\frac{(\sum x)^2}{n}$$
\end{prop}
\begin{defn}[Varianza muestral, fórmula breve]
$$s^2=\frac{{\sum_{i=1}^n x_i^2}-\frac{\left(\sum_{i_1}^n x_i\right)^2}{n}}{n-1}$$

Le llamamos fórmula breve porque evita el cálculo de la media.
\end{defn}
\end{frame}

\begin{frame}
\frametitle{Medidas de dispersión}
\begin{prop}
Seán $x_1,x_2,x_3,\ldots,x_n$ una muestra y $c$ cualquier constante diferente de cero.
\begin{itemize}
\item[\textbf{1.}]Si $y_1=x_1+c, y_2=x_2+c, \ldots, y_n=x_n+c$, entonces $s_x^2=s_y^2$ y
\item[\textbf{2.}]Si $y_1=cx_1, y_2=cx_2, \ldots, y_n=cx_n$, entonces $s_y^2=c^2s_x^2$, y también $s_y=cs_x$ 
\end{itemize} 
donde $s_x^2$ es la varianza muestral de las $x$ y $s_y^2$ es la varianza muestral de las $y$.
\end{prop}
En otras palabras, el resultado 1 dice que si se suma (o se resta) una constante $c$ de cada valor de dato, la varianza no cambia. Esto es intuitivo puesto que la adición o sustracción de $c$ cambia la ubicación del conjunto de datos, pero deja inalteradas las distancias entre los valores de datos. De acuerdo con el resultado 2, la multiplicación de cada $x_i$ por $c$ hace que $s^2$ sea multiplicada por un factor de $c^2$.
\end{frame}

\begin{frame}
\frametitle{Medidas de dispersión - Ejercicios}
\begin{itemize}
\item[1.]Dada la muestra 6, 8, 7, 5, 3, 7. Encuentre lo siguiente:
\begin{itemize}
\item[a.]Rango
\item[b.]Varianza $s^2$ utilizando ambas fórmulas
\item[c.]Desviación estándar $s$
\end{itemize}
\item[2.]Dada la muestra 7, 6, 10, 7, 5, 9, 3, 7, 5, 13. Encuentre lo siguiente:
\begin{itemize}
\item[a.]Rango
\item[b.]Varianza $s^2$ utilizando ambas fórmulas
\item[c.]Desviación estándar $s$
\end{itemize}
\item[3.]Cada una de dos muestras tiene una desviación estándar de 5. Si los dos conjuntos de datos se agrupan en un conjunto de 10 datos, ¿la nueva muestra tendrá una desviación estándar que sea menor, igual o mayor que la desviación estándar original de 5? Para justificar su respuesta, haga dos conjuntos de datos, cada uno con una desviación estándar de 5. Incluya los cálculos.
\end{itemize} 
\end{frame}

\begin{frame}
\frametitle{Medidas de dispersión - Ejercicios}
\begin{itemize}
\item[4.]Los siguientes resultados en sumatorias nos ayudarán a obtener la formula breve de la varianza muestral $s^2$. Para cualquier constante c,
\begin{itemize}
\item[\textbf{a}]$\sum_{i=1}^n c=nc$
\item[\textbf{b}]$\sum_{i=1}^n cx_i=c\sum_{i=1}^n x_i$
\item[\textbf{c}]$\sum_{i=1}^n (x_i+y_i)=\sum_{i=1}^n x_i+\sum_{i=1}^n y_i$
\end{itemize}
Use a, b y c para demostrar que
$$s^2=\frac{\sum_{i=1}^n (x_i-\bar{x})^2}{n-1}=\frac{{\sum_{i=1}^n x_i}-\frac{\left(\sum_{i=1}^n x_i\right)^2}{n}}{n-1}$$
\item[5.]Demuestra que
\begin{itemize}
\item[\textbf{1.}]Si $y_1=x_1+c, y_2=x_2+c, \ldots, y_n=x_n+c$, entonces $s_x^2=s_y^2$ y
\item[\textbf{2.}]Si $y_1=cx_1, y_2=cx_2, \ldots, y_n=cx_n$, entonces $s_y^2=c^2s_x^2$, y también $s_y=cs_x$ 
\end{itemize} 
donde $x_1,x_2,x_3,\ldots,x_n$ es una muestra y $c$ es una constante.
\end{itemize} 
\end{frame}


\end{document}
